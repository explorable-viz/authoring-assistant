\section{Linguistics}
For a starting point in our investigation into the (hopefully limited) side of linguistics we'll need to explore, I'm 
beginning by taking a look at \citet{poesio23}. It provides what seems like a comprehensive review treatment of anaphors
in computational linguistics. 

A coreference chain is a cluster of mentions, ie pronouns that all refer to the same entity. In our case, we might
expect that an explanation related to a visualization or computation comprises of a number of references to (parts of)
the same output. For example, when providing an explanation of a view, our expressions all refer to the same view, or
set of views. Unsure how to model parts of an object, but intuitively, it feels like a co-reference chain would
refer to the enclosing scope of their constituent parts, ie a \kw{LineChart} constructor is implicitly referred to
whenever we have an expression that relates to a component of the chart.