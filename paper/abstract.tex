\begin{abstract}
We introduce the idea of \emph{transparent text}, web-based data-driven scholarly articles which allow a
reader to explore the relationship to the data set by hovering over fragments of text, and present an
agent-based LLM framework for authoring transparent text. Our approach builds on recent developments in
integrated data provenance for general-purpose programming languages. Our system uses Fluid, an open source
programming language which is able to link visual elements to data using a provenance-tracking runtime, and
uses it as host platform for transparent text. Our architecture consists of two LLM agents which support the
human author during the creation of a transparent document. A SuggestionAgent identifies fragments of text
which could in principle be computed from data, including numerical values selected from records or computed
by aggregations like sum and mean, comparatives and superlatives like ``better than'' and ``largest'',
trend-adjectives like ``growing'', and other idiomatic quantitative or semi-quantitative phrases. An
InterpretationAgent synthesises suitable Fluid queries over the data, generating either a target string (when
the desired text fragment is already known), or a candidate string (when the answer to the query is not
known). The Fluid runtime infrastructure then interprets the resulting expression in the context of an
interactive web page which the author can use to manually validate the generated code. We evaluate our
approach on a subset of SciGen, a dataset consisting of tables from scientific articles and their
corresponding descriptions, which we extend with hand-generated counterfactual test cases for evaluating how
machine-generated expressions generalise in the presence of changes to the underlying data. Our results show
that some state-of-the-art models are able to synthesise compound expressions that are extensionally
compatible with gold solutions.
\end{abstract}
