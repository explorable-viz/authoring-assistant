\documentclass{article}

\usepackage{microtype}
\usepackage{graphicx}
\usepackage{subcaption}
\usepackage{booktabs}

% If your build breaks (sometimes temporarily if a hyperlink spans a page)
% please comment out the following usepackage line and replace
% \usepackage{icml2026} with \usepackage[nohyperref]{icml2026} above.
\usepackage{hyperref}

% Attempt to make hyperref and algorithmic work together better:
\newcommand{\theHalgorithm}{\arabic{algorithm}}

% Use the following line for the initial blind version submitted for review:
% \usepackage{icml2026}

% If accepted, instead use the following line for the camera-ready submission:
\usepackage[accepted]{icml2026}

\usepackage{amsmath}
\usepackage{amssymb}
\usepackage{mathtools}
\usepackage{amsthm}


% if you use cleveref..
\usepackage[capitalize,noabbrev]{cleveref}

%%%%%%%%%%%%%%%%%%%%%%%%%%%%%%%%
% THEOREMS
%%%%%%%%%%%%%%%%%%%%%%%%%%%%%%%%
\theoremstyle{plain}
\newtheorem{theorem}{Theorem}[section]
\newtheorem{proposition}[theorem]{Proposition}
\newtheorem{lemma}[theorem]{Lemma}
\newtheorem{corollary}[theorem]{Corollary}
\theoremstyle{definition}
\newtheorem{definition}[theorem]{Definition}
\newtheorem{assumption}[theorem]{Assumption}
\theoremstyle{remark}
\newtheorem{remark}[theorem]{Remark}

% The \icmltitle you define below is probably too long as a header.
% Therefore, a short form for the running title is supplied here:
\icmltitlerunning{AI-Assisted Authoring for Transparent, Data-Driven Documents}

\begin{document}

\twocolumn[
  \icmltitle{AI-Assisted Authoring for Transparent, \\Data-Driven Documents}

  % It is OKAY to include author information, even for blind submissions: the
  % style file will automatically remove it for you unless you've provided
  % the [accepted] option to the icml2026 package.

  % List of affiliations: The first argument should be a (short) identifier you
  % will use later to specify author affiliations Academic affiliations
  % should list Department, University, City, Region, Country Industry
  % affiliations should list Company, City, Region, Country

  % You can specify symbols, otherwise they are numbered in order. Ideally, you
  % should not use this facility. Affiliations will be numbered in order of
  % appearance and this is the preferred way.
  \icmlsetsymbol{equal}{*}

  \begin{icmlauthorlist}
    \icmlauthor{Alfonso Piscitelli}{equal,Salerno}
    \icmlauthor{Cristina David}{equal,Bristol}
    \icmlauthor{Mattia De Rosa}{Salerno}
    \icmlauthor{Ali Muhammad}{sch}
    \icmlauthor{Federico Nanni}{yyy}
    \icmlauthor{Jacob Pake}{sch,yyy,comp}
    \icmlauthor{Roly Perera}{comp}
    \icmlauthor{Jessy Sodimu}{sch}
    \icmlauthor{Chenyiqiu Zheng}{yyy,comp}
  \end{icmlauthorlist}

  \icmlaffiliation{Salerno}{Department of Computer Science, University of Salerno, Fisciano, Italy}
  \icmlaffiliation{Bristol}{School of Computer Science, University of Bristol, Bristol, UK}
  \icmlaffiliation{sch}{School of ZZZ, Institute of WWW, Location, Country}

  \icmlcorrespondingauthor{Alfonso Piscitelli}{apiscitelli@unisa.it}
  \icmlcorrespondingauthor{Roly Perera}{roly.perera@cl.cam.ac.uk}

%  \icmlkeywords{Machine Learning, ICML}

  \vskip 0.3in
]

% this must go after the closing bracket ] following \twocolumn[ ...

% This command actually creates the footnote in the first column listing the
% affiliations and the copyright notice. The command takes one argument, which
% is text to display at the start of the footnote. The \icmlEqualContribution
% command is standard text for equal contribution. Remove it (just {}) if you
% do not need this facility.

% Use ONE of the following lines. DO NOT remove the command.
% If you have no special notice, KEEP empty braces:
\printAffiliationsAndNotice{}  % no special notice (required even if empty)
% Or, if applicable, use the standard equal contribution text:
% \printAffiliationsAndNotice{\icmlEqualContribution}

\begin{abstract}
  This document provides a basic paper template and submission guidelines.
  Abstracts must be a single paragraph, ideally between 4--6 sentences long.
  Gross violations will trigger corrections at the camera-ready phase.
\end{abstract}

\subsection{Author Information for Submission}
\label{author info}

ICML uses double-blind review, so author information must not appear. If
you are using \LaTeX\/ and the \texttt{icml2026.sty} file, use
\verb+\icmlauthor{...}+ to specify authors and \verb+\icmlaffiliation{...}+
to specify affiliations. (Read the TeX code used to produce this document for
an example usage.) The author information will not be printed unless
\texttt{accepted} is passed as an argument to the style file. Submissions that
include the author information will not be reviewed.

\subsection{Citations and References}

Please use APA reference format regardless of your formatter or word processor.
If you rely on the \LaTeX\/ bibliographic facility, use \texttt{natbib.sty} and
\texttt{icml2026.bst} included in the style-file package to obtain this format.

Citations within the text should include the authors' last names and year. If
the authors' names are included in the sentence, place only the year in
parentheses, for example when referencing Arthur Samuel's pioneering work
\yrcite{Samuel59}. Otherwise place the entire reference in parentheses with the
authors and year separated by a comma \cite{Samuel59}. List multiple references
separated by semicolons \cite{kearns89,Samuel59,mitchell80}. Use the `et~al.'
construct only for citations with three or more authors or after listing all
authors to a publication in an earlier reference \cite{MachineLearningI}.

Authors should cite their own work in the third person in the initial version
of their paper submitted for blind review. Please refer to \cref{author info}
for detailed instructions on how to cite your own papers.

Use an unnumbered first-level section heading for the references, and use a
hanging indent style, with the first line of the reference flush against the
left margin and subsequent lines indented by 10 points. The references at the
end of this document give examples for journal articles \cite{Samuel59},
conference publications \cite{langley00}, book chapters \cite{Newell81}, books
\cite{DudaHart2nd}, edited volumes \cite{MachineLearningI}, technical reports
\cite{mitchell80}, and dissertations \cite{kearns89}.



\bibliography{tex-common/bib}
\bibliographystyle{icml2026}

\newpage
\appendix
\onecolumn
\section{You \emph{can} have an appendix here.}

\end{document}
