\begin{table}[!ht]
    \centering
    \footnotesize
    \renewcommand{\arraystretch}{1.2}
    \begin{tabular}{>{\raggedright\arraybackslash}p{2cm} >{\raggedright\arraybackslash}p{5cm} >{\raggedright\arraybackslash}p{6cm}}
        \hline
        \textbf{Type}                & \textbf{Example} & \textbf{Gold Solution} \\
        \rowcolor{gray!20}
        \multicolumn{3}{>{\raggedright\arraybackslash}l}{\textbf{Quantitative expressions}} \\

        Numerical value
        & the training time per epoch growing from \hl{67} seconds to 106 seconds.
        &
        \begin{lstlisting}[language=Fluid,numbers=none,aboveskip=-7pt,belowskip=-8.5pt]
(findWithKey_ "model" "LSTM" tableData).time_s
        \end{lstlisting}
        \\
        Percentage &
        The Energy Sector accounts for total methane emissions of \hl{52.80}\% in 2030.
        &
        \begin{lstlisting}[language=Fluid,numbers=none,aboveskip=-7pt,belowskip=-8.5pt]
(record.emissions /
 sum (map (fun x -> x.emissions)
          (getByYear year tableData))) * 100

        \end{lstlisting}  \\
        \rowcolor{gray!20}
        \multicolumn{3}{>{\raggedright\arraybackslash}l}{\textbf{Aggregation}} \\
        Average
        & The average methane emissions for the year 2030 is \hl{13.51} &
        \begin{lstlisting}[language=Fluid,numbers=none,aboveskip=-7pt,belowskip=-8.5pt]
(sumEmissions year tableData / length records)
        \end{lstlisting} \\
        Min/Max                          & The Energy Sector recorded its highest methane emissions in \hl{2030}             &
        \begin{lstlisting}[language=Fluid,numbers=none,aboveskip=-7pt,belowskip=-8.5pt]
let maxEntry = maximumBy
   (fun x -> x.emissions)
   (filter (fun x -> x.type == "Energy Sector")
      tableData)
in maxEntry.year
        \end{lstlisting} \\                             \\
        Rank &
        3-layer stacked CNN gives an accuracy of 81.46\%, which is the \hl{lowest} compared with BiLSTM, and S-LSTM  &
        \begin{lstlisting}[language=Fluid,numbers=none,aboveskip=-7pt,belowskip=-8.5pt]
let pos = findIndex "model" "CNN"
    (insertionSort cmpTime tableData)
in rankLabel "lowest" pos \end{lstlisting} \\
        Total &
        The total methane emissions for the year 2030 is \hl{37.74} for Agriculture &
        \begin{lstlisting}[language=Fluid,numbers=none,aboveskip=-7pt,belowskip=-8.5pt]
sumEmissions year tableData
        \end{lstlisting} \\
        \rowcolor{gray!20}
        \multicolumn{3}{>{\raggedright\arraybackslash}l}{\textbf{Trends}} \\

        Comparison
        & The training time per epoch \hl{growing} from 67 seconds to 106 seconds. &
        \begin{lstlisting}[language=Fluid,numbers=none,aboveskip=-7pt,belowskip=-8.5pt]
trendWord
 (findWithKey_ "model" "BiLSTM" tableData).time_s
 (findWithKey_ "model" "LSTM" tableData).time_s
 growShrink
        \end{lstlisting} \\
        Pairwise comparisons
        & Overall the SVM with polynomial degree 1 kernel outperformed all other kernels with other kernels generally offering \hl{better} precision at a higher cost to recall. &
        \begin{lstlisting}[language=Fluid,numbers=none,aboveskip=-7pt,belowskip=-8.5pt]
betterWorse (overallComparison 
  [compareToSvmLinearP c 
  | c <- ["naive_bayes_p", "svm2_p", "svm3_p", "svmr_p"]]
)

        \end{lstlisting} \\
        \hline
    \end{tabular}
    \caption{Quantitative/semi-quantitative natural language forms considered in this paper \mdnote{In this table, the examples are written in Fluid. There are references to Fluid in the introduction~\citep{perera22,bond25}, but without reading them it may be difficult to understand. Would it be worth briefly explaining Fluid here as well, to make the paper more self-contained (assuming there is space for it)?}}
    \label{tab:natural-language-forms}
\end{table}
