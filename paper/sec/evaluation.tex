\section{Experimental Evaluation}
\label{sec:evaluation}

\subsection{Research questions}

RQ1. How robust is the system when provided with incomplete or ambiguous information?
In real-world authoring tasks, text fragments may omit key criteria (e.g. \textbf{?}) or contain vague identifiers (e.g. \textbf{?} without further context).
We therefore ask how performance degrades when the specification is underspecified or when multiple interpretations are possible. Here, we distinguish between two forms of ambiguity:
(1) domain-level ambiguity, which reflects genuine uncertainty in the data (e.g. two models achieving identical scores).
(2) specification-level ambiguity, which arises from incomplete or imprecise descriptions in the query (e.g. missing fields that would otherwise resolve the expression uniquely).

RQ2. How sensitive is the system to misleading or adversarial information?
Authors may, intentionally or unintentionally, introduce misleading elements into the text. These include inconsistent identifiers, incorrect numerical values, or contradictory claims.
We ask how the system responds in such cases: does it fail gracefully, does it default to literal reproduction of the misleading fragment, or does it attempt to generate a plausible, but incorrect, expression?

Together, these questions probe the limits of the InterpretationAgent's ability to generate reliable code. By contrasting genuine uncertainty with misleading or ill-specified inputs, we aim to understand not only the model's raw success rate, but also the failure modes most likely to arise in practical use.

\subsection{Success Rate by Category}
To evaluate the system we used a sample of the \SciGen dataset~\citep{moosavi21}. We aggregated the generated
expression in three categories: aggregation, trends, quantitative (See
Table~\ref{tab:natural-language-forms}). \figref{success_rate_by_category} shows the results obtained in
generating expression by categories.

\begin{figure}
    \centering
    \begin{subfigure}{0.48\linewidth}
        \centering
        \includegraphics[width=\linewidth]{fig/success_rate_by_category}
        \caption{Success rate by Linguistic Category}
        \label{fig:success_rate_by_category}
    \end{subfigure}\hfill
    \begin{subfigure}{0.48\linewidth}
        \centering
        \includegraphics[width=\linewidth]{fig/success_rate_by_complexity}
        \caption{Success rate by Complexity}
        \label{fig:success_rate_by_complexity}
    \end{subfigure}
    \caption{Success rate of the proposed system, measured over 5 runs with \gptfour}
    \label{fig:success_rate_comparison}
\end{figure}

\subsection{Results}

\begin{table}[h!]
    \scriptsize

    \centering
    \begin{subtable}{0.45\linewidth}
        \centering
        \begin{tabular}{lcccc}
            \hline
            & \multicolumn{2}{c}{\textbf{\gptfour}} & \multicolumn{2}{c}{\textbf{\gptfive}} \\
            & \textbf{Success Rate (\%)} & \textbf{SD (\%)} & \textbf{Success Rate} & \textbf{SD (\%)} \\
            \hline
            Comparison     & 68.2   & 3.2 & 77.4 & 2 \\
            Data Retrieval & 77.3   & 3.7 & 77.4 & 2 \\
            Difference     & 20.0   & 44.7 & 77.4 & 2 \\
            Min/max        & 97.0   & 5.6 & 77.4 & 2 \\
            Rank           & 0.0   & 0.0 & 77.4 & 2 \\
            Sum            & 100   & 0.0 & 77.4 & 2 \\
            Ratio          & 100   & 0.0 & 77.4 & 2 \\
            Agerage        & 100   & 0.0 & 77.4 & 2 \\
            Overall        & 74.9 & 3 & 77.4 & 2 \\
            \hline
        \end{tabular}
        \caption{Results with target-value = true}
        \label{tab:results_true}
    \end{subtable}%
    \hfill
    \begin{subtable}{0.45\linewidth}
        \centering
        \begin{tabular}{lcccc}
            \hline
            & \multicolumn{2}{c}{\textbf{\gptfour}} & \multicolumn{2}{c}{\textbf{\gptfive}} \\
            & \textbf{Success Rate (\%)} & \textbf{SD (\%)} & \textbf{Success Rate} & \textbf{SD (\%)} \\
            \hline
            Comparison     & 33.3 & 3.1 & -- & -- \\
            Data Retrieval & 62.0 & 1.6 & -- & -- \\
            Difference     & 75.0 & 50.0 & -- & -- \\
            Min/max        & 90.6 & 12.6 & -- & -- \\
            Rank           & 0.0 & 0.0 & -- & -- \\
            Sum            & 100.0 & 0.0 & -- & -- \\
            Ratio          & 100.0 & 0.0 & -- & -- \\
            Average        & 100.0 & 0.0 & -- & -- \\
            Overall        & 57.1 & 0.8 & -- & -- \\
            \hline
        \end{tabular}
        \caption{Results without target-value}
        \label{tab:results_no_target}
    \end{subtable}
    \caption{Success rates (with standard deviation) for \textbf{\gptfour} and \textbf{\gptfive} across different task categories}
    \label{tab:results_combined}
\end{table}

In this section, we report the experimental results.
Table \ref{tab:results_combined} summarizes the performance of \gptfour and the most recent OpenAI model, \gptfive, under two different settings: with target-value sharing enabled and without target-value sharing.
Overall, the system produced a valid Fluid expression in 74.9\% of the cases (S.D. 3\%).

Figure~\ref{fig:success_rate_by_category} illustrates the success rate for each category, both with and without target-value sharing.
The results show that providing the target value is crucial for achieving a high success rate.
Moreover, the \InterpretationAgent generates a correct expression in approximately XX\% of the cases.

The system achieved performance above 70\% in comparison, data retrieval, and min/max search tasks.
In contrast, accuracy decreased significantly in expressions requiring differences (20\%) and in ranking tasks (50\%).

Figure~\ref{fig:success_rate_by_complexity} reports the success rate as a function of the number of categories assigned to each expression.
Fluid instructions combining multiple operations were associated with multiple categories.
Results show that the success rate was 75\% for simple expressions (one category), increased to 85\% for two categories,
and dropped to 58\% when three categories were involved.

In addition to evaluating performance when the target expression is provided, we also assessed the model’s ability to produce a
correct expression when no reference string was given.
On the \SciGen dataset, the model achieved a 74.9\% success rate when provided with the target string.
Without the target string, performance decreased moderately to 39.4\%, indicating that the model retains substantial
ability to reconstruct valid solutions even without explicit guidance

\subsection{Counterfactual testing}
\label{sec:validation-with-counterfactual-data}

For the \emph{counterfactual testing} part of our analysis, the input data is deliberately modified in order
to verify the consistency of the behaviours. Both the expected and generated expressions are executed again
and the results are compared. If discrepancies are found the \InterpretationAgent ask for a new generation
sending back that the failure is related to a counterfactual error.

%\rpnote{complete}

Across 300 test executions, 121 contained at least one counterfactual error (avg. 3.8), of which 42 ultimately succeeded.
Such tests help detect expressions that may yield the expected output but not the correct solution.
For example, the \texttt{\textbackslash InterpretationAgent} generated:

\begin{verbatim}
(findWithKey_ "model" "LSTM2" tableData).time_s
\end{verbatim}

intended to retrieve the execution time of the \texttt{LSTM} model, but incorrectly referred to \texttt{LSTM2}.
Counterfactual testing exposed this error and prompted regeneration.


\subsection{Problematic Cases}
\label{subsec:problematic-cases}

Table~\ref{tab:problematic_cases} presents examples of problematic cases in which the system was asked to generate a valid
Fluid expression.
In these cases, producing a valid expression is challenging because the input data (or the NL query) contains ambiguities
that prevent the construction of a well-formed expression.

For instance, in the \textit{missing-data} problem, the \InterpretationAgent should generate a function to compute
 the average of the emissions. However, the input data does not provide information for all months.

In such cases, the system was unable to generate a general expression and instead produced the expected value directly.

\begin{table}[t]
    \centering
    \small
    \begin{tabular}{l p{6cm} p{4cm}}
        \hline
        \textbf{Problem Type} & \textbf{Example} & \textbf{Explanation} \\
        \hline

        ambiguous referent &
        As shown in Table 3 \hl{BiLSTM} gives significantly better accuracies compared to uni-directional LSTM with the training time per epoch growing from 99 seconds to 106 seconds &
        There are three models that have 99 seconds as their time; the assistant would not know which one to choose. \\

        missing data &
        The average methane emissions for the year 2015 is \hl{1.0} &
        November is missing to calculate the average correctly. \\

        false statement &
        We additionally make comparisons with stacked CNNs and hierarchical attention (Vaswani et al., 2017), shown in Table 3 (the CNN and Transformer rows), BiLSTM is \hl{the most efficient} among all models compared, with the highest model size &
        BiLSTM is not the most efficient, nor does it have the largest size. \\

        incorrect numerical value &
        LSTM is the fastest model with overall time taken being \hl{90} seconds &
        It is not 90 but 106. \\

        ambiguous referent &
        LSTM is the fastest model with overall time taken being \hl{90} seconds &
        There are two type of time in the dataset (training\_time, execution\_time), both with a value of 90 seconds. \\

        \hline
    \end{tabular}
    \caption{Categories of problematic example}
    \label{tab:problematic_cases}
\end{table}
