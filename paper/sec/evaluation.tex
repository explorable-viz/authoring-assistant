\section{Experimental Evaluation}
\label{sec:evaluation}

\subsection{Research questions}

Our evaluation tests the ability of the \InterpretationAgent to translate quantitative and semi-quantitative expressions
from scholarly natural language into executable queries that operate on data. Beyond raw accuracy, we are also concerned
with how performance varies with task complexity, and whether the generated expressions are robust under changes to data
or in the presence of ambiguity or other low data quality issues. These are captured in two research questions:

\paragraph{RQ1. Interpretation Accuracy across Query Categories and Complexity.}

To what extent can LLMs accurately interpret quantitative and semi-quantitative claims in scholarly text as data
queries? We examine performance across the various ``query categories'' summarised in
Table~\ref{tab:natural-language-forms} (namely averages, percentages, min/max, rank, etc), and investigate how accuracy
varies with task complexity --- measured, somewhat crudely, by the cardinality of the multiset of query categories
present in the gold solution.

\paragraph{RQ2. Generalisability and Robustness.}

How well do the generated expressions generalise when the underlying data changes, or when the input contains
misleading or ill-specified information? We test whether generated queries continue to produce correct outputs
under a set of hand-generated counterfactual modifications of the dataset, based on expected query results
specific to each test case, and also how counterfactual performance is impacted by the presence of misleading
or adversarial phrasing.

\begin{figure*}
    \centering
    \begin{subfigure}{0.48\linewidth}
        \centering
        \includegraphics[width=\linewidth]{fig/benchmark/gpt-4o/scigen-manual-old/success_rate_by_category_boxplot}
        \caption{Success rate by Linguistic Category}
        \label{fig:success_rate_by_category}
    \end{subfigure} \hfill
    \begin{subfigure}{0.48\linewidth}
        \centering
        \includegraphics[width=\linewidth]{fig/benchmark/gpt-4o/scigen-manual-old/success_rate_by_complexity}
        \caption{Success rate by Complexity}
        \label{fig:success_rate_by_complexity}
    \end{subfigure}
    \caption{Success rate of the proposed system, measured over 5 runs with \gptfour}
    \label{fig:success_rate_comparison}
\end{figure*}

\subsection{Results}

\paragraph{Interpretation Accuracy across Linguistic Idioms and Complexity.} To evaluate RQ1, we used a sample
of the \SciGen dataset~\citep{moosavi2021}, an open source dataset consisting of tables from scientific articles
and their corresponding descriptions. We aggregated the results according to the linguistic categories from
Table~\ref{tab:natural-language-forms}. Figure~\ref{fig:success_rate_by_category} illustrates the success rate
for each category, both with and without target-value sharing.

The results show that \emph{the system is robust when provided with sufficient guidance but degrades when
underspecified.} With the target-value sharing, the InterpretationAgent produced correct Fluid expressions in
74.9\%  (S.D. 3.0\%) of cases, but performance dropped to 57.1\% when the target was withheld. This highlights
the system's reliance on explicit cues when resolving ambiguous fragments.

\emph{Performance also varied across linguistic categories.} Success rates exceeded 68\% for comparison,
77.3\% for data retrieval, and 97\% for min/max search tasks. In contrast, accuracy decreased significantly
for expressions requiring differences (20\%) and for ranking tasks (0\%).

The trend for compositional complexity is more nuanced as shown in
Figure~\ref{fig:success_rate_by_complexity}, which reports the success rate as a function of the number of
categories assigned to each expression: success rates are 62\% for single-category expressions, increase to
91\% when two categories are combined, but collapse to 0\% when three categories are involved. This suggests
that \emph{moderate composition can actually aid performance, perhaps by giving the model clearer structural
cues, but that complexity beyond a certain threshold overwhelms the synthesis process}.

\paragraph{Generalisability and Robustness.} As a preliminary attempt to address RQ2, we carried out
\emph{counterfactual testing} to evaluate the robustness of generated expressions under changes to the
underlying data. In this setup, the input tables were modified according to hand-craft test specifications,
and both the expected and generated expressions were re-executed to check whether the behaviours remained
consistent. Across 300 test executions, 121 contained at least one counterfactual error (an average of 3.8 per
case), of which 42 ultimately still succeeded. These tests highlight cases where an expression may
coincidentally yield the correct output on the original data but fails to be extensionally equivalent more
generally (i.e.~under perturbation). For example, in one test the system generated

\hspace{3mm} \kw{(findWithKey\_ "model" "LSTM2" tableData).time\_s}

intended to retrieve the execution time of the \texttt{LSTM} model, but incorrectly referred to \texttt{LSTM2}.
Counterfactual testing exposed this mismatch, which would otherwise have gone undetected.

At present, counterfactual tests are used only as an evaluation device, not as part of the authoring workflow
itself. For future work (\Secref{conclusion}), we plan to investigate automatic generation of counterfactual
tests, allowing these additional robustness checks to be integrated into the document authoring workflow.

% \begin{table*}[t]
    \centering
    \small
    \begin{tabular}{l p{6cm} p{4cm}}
        \hline
        \textbf{Problem Type} & \textbf{Example} & \textbf{Explanation} \\
        \hline
        false comparison &
        BiLSTM is \hl{the most efficient} among all models compared, with the highest model size &
        BiLSTM is not the most efficient, nor does it have the largest size. \\

        wrong numerical value &
        LSTM is the fastest model with overall time taken being \hl{90} seconds &
        It is not 90 but 106. \\

        ambiguous referent &
        LSTM is the fastest model with overall time taken being \hl{90} seconds &
        There are two type of time in the dataset (training\_time, execution\_time), both with a value of 90 seconds. \\

        \hline
    \end{tabular}
    \caption{Categories of problematic example}
    \label{tab:problematic_cases}
\end{table*}

