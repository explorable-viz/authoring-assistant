\section{Introduction: Transparent, Data-Driven Documents}

Making sense of and verifying data-driven claims is hard even with open and accessible data. A key challenge
lies in tracing specific claims back to the relevant data. In peer review, for example, empirical claims
typically lack author-supplied links to data, making them hard for reviewers to check
directly~\citep{weber20}. Paper retractions, meanwhile, are often attributable not to fraud, but to simple
errors in data management or analysis~\citep{hu25}. The use of large language models (LLMs) to interpret
scholarly documents has seen considerable attention recently, from fact-checking~\citep{abu-ahmad25} to
interpretation of charts and figures~\citep{roberts24}, but current LLM interfaces do not support direct
interrogation of visual or other outputs for traceability to inputs.

Recent advances in data provenance and data visualisation~\citep{psallidas18smoke,bond25}, on the other hand,
have pushed in this direction using a more infrastructural approach. These approaches link computed outputs
to their data sources directly by tracking dependency information. This allows visual outputs to support
\emph{provenance queries}, user interactions (e.g. mousing over visual elements) that reveal how output
features relate to data. The advantage of this approach is that the relationships to data sources are exposed
automatically via trusted infrastructure, typically a query language or general-purpose programming language
which tracks how data flows through a computation. However, these approaches are limited to outputs computed
from data, such as visualisations. What is missing is a way to extend these ``direct interrogation'' features
to natural language itself, where the main claims of most scholarly articles are actually made.

In this paper, we address this gap by combining two complementary approaches: the ability of LLMs to
understand technical language and synthesise queries over data, plus the provenance-tracking infrastructure of
an open source programming language called Fluid (\url{https://f.luid.org/})~\citep{perera22,bond25}.
Together, these two technologies enable the creation of \emph{transparent documents}, web-based scholarly
articles with two key transparency features:
\begin{enumerate}
\item \textbf{Data-driven:} Quantitative statements expressed in natural language --- e.g.~that system $X$ is
faster than system $Y$ on some task --- are computed from the relevant data, rather than occurring
merely as static fragments of text.

\item \textbf{Data linking:} Readers and reviewers can interactively trace such claims back to the specific
data elements that support them, through embedded provenance queries.
\end{enumerate}

\figref{scigen-example-website}, generated from our implementation, illustrates these two features. The upper
section shows a ``transparent'' excerpt from \cite{zhang18}, a scholarly article comparing text encoding
techniques. When a reader hovers over the phrase \textfrag{does not further improve}, the relevant data are
highlighted on the left. Other fragments (e.g. \textfrag{better than}, \textfrag{further improvements}) that
refer to the same data are also marked, allowing the reader to explore supporting and contrasting evidence.
The lower section shows a counterfactual situation where the authors' experiments had produced different
results: here the phrase \textfrag{does not further improve} is replaced by \textfrag{further improves}.

\begin{figure}%[h]
    \centering
    \includegraphics[width=\linewidth]{fig/scigen-1805.02474v1-10-with-pointer.png}
    \vspace{1mm}
    \hrule
    \includegraphics[width=\linewidth]{fig/scigen-1805.02474v1-10-counterfactual-with-pointer.png}
    \caption{\emph{Transparent document} example, showing data-driven natural language (top
    vs.~bottom) \mdnote{The caption could perhaps say a little more about the figure, although it is already described in the text.}}
    \label{fig:scigen-example-website}
\end{figure}

This transparent version of the document was implemented in Fluid. The source code is shown in
\figref{fluid-example-paragraph}, and makes use of several helper functions, a representative subset of which
are shown in \figref{fluid-scigen}. Both the provenance-tracking runtime of Fluid \emph{and} the LLM-based
authoring support contribute to the solution: the Fluid runtime provides the interactions, and the LLM-based
authoring tool makes the whole process practical. Without tool support, authoring a scientific document as a
(hand-crafted) \emph{program} is unlikely to be a feasible task.

AI-assisted authoring of transparent documents thus support turning static text into interactable, data-driven
content able to expose the evidential basis of scholarly claims. We envisage two use cases. First, when
\textbf{authoring} content for an online article, a journalist or scientific publisher may wish to provide
text which is linked to the underlying data so that the evidence base for the textual claims can be explored
directly from the article. Second, when \textbf{reading} a document reporting on findings derived from open
data (perhaps a scientific paper or climate report), the reader may want to retroactively interpret parts of
the text as queries over the available data and gradually ``rationally reconstruct'' the relationship between
claims in the paper and the evidence base. This might be just to aid their own comprehension, or part of a
formal peer review process.

\paragraph{Contributions.} Our specific contributions are:

\begin{itemize}
\item A proof-of-concept LLM-based tool for iteratively transforming a preexisting opaque document and
associated data set into a transparent, data-driven counterpart;
\item An empirical evaluation of how well state-of-the-art models are able to solve the associated
interpretation and code synthesis problems.
\end{itemize}

We leave implementing a full Copilot-like authoring plugin for an IDE such as VSCode or Cursor for future
work.

\begin{figure}[h]
    \small
    % listings package seems to struggle, for now just use screenshot from VSCode
    \includegraphics[scale=0.169]{fig/scigen-1805.02474v1-10-src-screenshot.png}
    \caption{Gold solution for transparent text in \figref{scigen-example-website}}
    \label{fig:fluid-example-paragraph}
\end{figure}



\subsection{Target idioms of natural language}
\cdnote{Can we make this a section? intro is getting too long}

\begin{table}[!ht]
    \centering
    \footnotesize
    \renewcommand{\arraystretch}{1.2}
    \begin{tabular}{>{\raggedright\arraybackslash}p{2cm} >{\raggedright\arraybackslash}p{5cm} >{\raggedright\arraybackslash}p{6cm}}
        \hline
        \textbf{Type}                & \textbf{Example} & \textbf{Gold Solution} \\
        \rowcolor{gray!20}
        \multicolumn{3}{>{\raggedright\arraybackslash}l}{\textbf{Quantitative expressions}} \\

        Numerical value
        & the training time per epoch growing from \hl{67} seconds to 106 seconds.
        &
        \begin{lstlisting}[language=Fluid,numbers=none,aboveskip=-7pt,belowskip=-8.5pt]
(findWithKey_ "model" "LSTM" tableData).time_s
        \end{lstlisting}
        \\
        Percentage &
        The Energy Sector accounts for total methane emissions of \hl{52.80}\% in 2030.
        &
        \begin{lstlisting}[language=Fluid,numbers=none,aboveskip=-7pt,belowskip=-8.5pt]
(record.emissions /
 sum (map (fun x -> x.emissions)
          (getByYear year tableData))) * 100

        \end{lstlisting}  \\
        \rowcolor{gray!20}
        \multicolumn{3}{>{\raggedright\arraybackslash}l}{\textbf{Aggregation}} \\
        Average
        & The average methane emissions for the year 2030 is \hl{13.51} &
        \begin{lstlisting}[language=Fluid,numbers=none,aboveskip=-7pt,belowskip=-8.5pt]
(sumEmissions year tableData / length records)
        \end{lstlisting} \\
        Min/Max                          & The Energy Sector recorded its highest methane emissions in \hl{2030}             &
        \begin{lstlisting}[language=Fluid,numbers=none,aboveskip=-7pt,belowskip=-8.5pt]
let maxEntry = maximumBy
   (fun x -> x.emissions)
   (filter (fun x -> x.type == "Energy Sector")
      tableData)
in maxEntry.year
        \end{lstlisting} \\                             \\
        Rank &
        3-layer stacked CNN gives an accuracy of 81.46\%, which is the \hl{lowest} compared with BiLSTM, and S-LSTM  &
        \begin{lstlisting}[language=Fluid,numbers=none,aboveskip=-7pt,belowskip=-8.5pt]
let pos = findIndex "model" "CNN"
    (insertionSort cmpTime tableData)
in rankLabel "lowest" pos \end{lstlisting} \\
        Total &
        The total methane emissions for the year 2030 is \hl{37.74} for Agriculture &
        \begin{lstlisting}[language=Fluid,numbers=none,aboveskip=-7pt,belowskip=-8.5pt]
sumEmissions year tableData
        \end{lstlisting} \\
        \rowcolor{gray!20}
        \multicolumn{3}{>{\raggedright\arraybackslash}l}{\textbf{Trends}} \\

        Comparison
        & The training time per epoch \hl{growing} from 67 seconds to 106 seconds. &
        \begin{lstlisting}[language=Fluid,numbers=none,aboveskip=-7pt,belowskip=-8.5pt]
trendWord
 (findWithKey_ "model" "BiLSTM" tableData).time_s
 (findWithKey_ "model" "LSTM" tableData).time_s
 growShrink
        \end{lstlisting} \\
        Pairwise comparisons
        & Overall the SVM with polynomial degree 1 kernel outperformed all other kernels with other kernels generally offering \hl{better} precision at a higher cost to recall. &
        \begin{lstlisting}[language=Fluid,numbers=none,aboveskip=-7pt,belowskip=-8.5pt]
betterWorse (overallComparison 
  [compareToSvmLinearP c 
  | c <- ["naive_bayes_p", "svm2_p", "svm3_p", "svmr_p"]]
)

        \end{lstlisting} \\
        \hline
    \end{tabular}
    \caption{Quantitative/semi-quantitative natural language forms considered in this paper \mdnote{In this table, the examples are written in Fluid. There are references to Fluid in the introduction~\citep{perera22,bond25}, but without reading them it may be difficult to understand. Would it be worth briefly explaining Fluid here as well, to make the paper more self-contained (assuming there is space for it)?}}
    \label{tab:natural-language-forms}
\end{table}


Table~\ref{tab:natural-language-forms} summarises the natural language idioms studied in this paper. With
state-of-the-art models like \gptfour and \gptfive, our system is able to resolve basic table lookups of direct
numerical values, as well as computations of percentages, averages, minima and maxima, and totals, each mapped
to the corresponding aggregation over the source data. For example, phrases such as \textfrag{the Energy
Sector accounts for 52.80\% of total emissions} and \textfrag{average methane emissions for 2030 is 13.51} are
interpreted in terms of sum and mean respectively over the relevant data values. Similarly, \textfrag{recorded
its highest emissions in 2030} is interpreted as a \kw{maximumBy} query, while a statement such as
\textfrag{CNN gives the lowest accuracy} is mapped to an explicit computation of rank.

We also consider \emph{trend} expressions, which comparative natural language phrases describing how a data
attribute evolves over time, such as \textfrag{training time growing from 67 to 106 seconds}. Such idioms are
mapped to higher-order functions like \kw{trendWord} parameterised on additional helper functions such as
\kw{growShrink} and \kw{betterWorse} (shown in \figref{fluid-scigen}) which map comparisons to appropriate
natural language phrases.

\begin{figure}%[h]
    \small
    {\lstinputlisting[language=Fluid]{../fluid-common/scigen.fld}}
    \caption{Sample of helper functions used in SciGen case study}
    \label{fig:fluid-scigen}
\end{figure}


Taken together, these categories cover a representative portion of the numerical reasoning idioms found in the
\SciGen benchmark. However, some linguistic forms that commonly arise in scholarly articles are not covered in
our analysis. We have yet to study approximate quantitative terms like \textfrag{around 50\%} or
\textfrag{roughly 100 instances}, nor interval-based descriptions such as \textfrag{between 30 and 40\%} or
\textfrag{within 5–10 seconds}. While we have no reason for thinking these will present specific difficulties,
other forms are likely to be more challenging. So-called \emph{graded} modal adverbs~\citep{lassiter17} which
modify adjectival comparatives like \textfrag{better} -- as in \textfrag{slightly better} and
\textfrag{significantly higher} -- especially when combined with trends over time, as in \textfrag{steadily
increasing} or \textfrag{sharply declining} -- are likely to prove difficult because the interpretation of
these qualifiers can be subjective and context-dependent. Generalised quantifiers like \textfrag{generally}
and \textfrag{usually}~\citep{barwise81} present similar challenges because colloquial use may differ from
more formal uses (in some situations ``most'' might mean a majority, i.e.~greater than 50\% of cases, but in
others may mean only ``greater than any other alternative proportion''). On the other hand these difficulties
also present themselves to human readers, so extending coverage to these idioms would substantially deepen our
tool's ability to bridge natural language reporting with interpretation in terms of the underlying dataset,
perhaps revealing inconsistent use of technical language on the part of the author. We discuss this further in
\secref{conclusion:future-work}.
