\section{Target idioms of natural language}
\label{sec:nl-idioms}

\begin{table}[!ht]
    \centering
    \footnotesize
    \renewcommand{\arraystretch}{1.2}
    \begin{tabular}{>{\raggedright\arraybackslash}p{2cm} >{\raggedright\arraybackslash}p{5cm} >{\raggedright\arraybackslash}p{6cm}}
        \hline
        \textbf{Type}                & \textbf{Example} & \textbf{Gold Solution} \\
        \rowcolor{gray!20}
        \multicolumn{3}{>{\raggedright\arraybackslash}l}{\textbf{Quantitative expressions}} \\

        Numerical value
        & the training time per epoch growing from \hl{67} seconds to 106 seconds.
        &
        \begin{lstlisting}[language=Fluid,numbers=none,aboveskip=-7pt,belowskip=-8.5pt]
(findWithKey_ "model" "LSTM" tableData).time_s
        \end{lstlisting}
        \\
        Percentage &
        The Energy Sector accounts for total methane emissions of \hl{52.80}\% in 2030.
        &
        \begin{lstlisting}[language=Fluid,numbers=none,aboveskip=-7pt,belowskip=-8.5pt]
(record.emissions /
 sum (map (fun x -> x.emissions)
          (getByYear year tableData))) * 100

        \end{lstlisting}  \\
        \rowcolor{gray!20}
        \multicolumn{3}{>{\raggedright\arraybackslash}l}{\textbf{Aggregation}} \\
        Average
        & The average methane emissions for the year 2030 is \hl{13.51} &
        \begin{lstlisting}[language=Fluid,numbers=none,aboveskip=-7pt,belowskip=-8.5pt]
(sumEmissions year tableData / length records)
        \end{lstlisting} \\
        Min/Max                          & The Energy Sector recorded its highest methane emissions in \hl{2030}             &
        \begin{lstlisting}[language=Fluid,numbers=none,aboveskip=-7pt,belowskip=-8.5pt]
let maxEntry = maximumBy
   (fun x -> x.emissions)
   (filter (fun x -> x.type == "Energy Sector")
      tableData)
in maxEntry.year
        \end{lstlisting} \\                             \\
        Rank &
        3-layer stacked CNN gives an accuracy of 81.46\%, which is the \hl{lowest} compared with BiLSTM, and S-LSTM  &
        \begin{lstlisting}[language=Fluid,numbers=none,aboveskip=-7pt,belowskip=-8.5pt]
let pos = findIndex "model" "CNN"
    (insertionSort cmpTime tableData)
in rankLabel "lowest" pos \end{lstlisting} \\
        Total &
        The total methane emissions for the year 2030 is \hl{37.74} for Agriculture &
        \begin{lstlisting}[language=Fluid,numbers=none,aboveskip=-7pt,belowskip=-8.5pt]
sumEmissions year tableData
        \end{lstlisting} \\
        \rowcolor{gray!20}
        \multicolumn{3}{>{\raggedright\arraybackslash}l}{\textbf{Trends}} \\

        Comparison
        & The training time per epoch \hl{growing} from 67 seconds to 106 seconds. &
        \begin{lstlisting}[language=Fluid,numbers=none,aboveskip=-7pt,belowskip=-8.5pt]
trendWord
 (findWithKey_ "model" "BiLSTM" tableData).time_s
 (findWithKey_ "model" "LSTM" tableData).time_s
 growShrink
        \end{lstlisting} \\
        Pairwise comparisons
        & Overall the SVM with polynomial degree 1 kernel outperformed all other kernels with other kernels generally offering \hl{better} precision at a higher cost to recall. &
        \begin{lstlisting}[language=Fluid,numbers=none,aboveskip=-7pt,belowskip=-8.5pt]
betterWorse (overallComparison 
  [compareToSvmLinearP c 
  | c <- ["naive_bayes_p", "svm2_p", "svm3_p", "svmr_p"]]
)

        \end{lstlisting} \\
        \hline
    \end{tabular}
    \caption{Quantitative/semi-quantitative natural language forms considered in this paper \mdnote{In this table, the examples are written in Fluid. There are references to Fluid in the introduction~\citep{perera22,bond25}, but without reading them it may be difficult to understand. Would it be worth briefly explaining Fluid here as well, to make the paper more self-contained (assuming there is space for it)?}}
    \label{tab:natural-language-forms}
\end{table}


Table~\ref{tab:natural-language-forms} summarises the natural language idioms studied in this paper. With
state-of-the-art models like \gptfour and \gptfive, our system is able to resolve basic table lookups of direct
numerical values, as well as computations of percentages, averages, minima and maxima, and totals, each mapped to the
corresponding aggregation over the source data. For example, the (highlighted) numerical quantities in phrases such as
\textfrag{the Energy Sector accounts for \hl{52.80}\% of total emissions} and \textfrag{average methane emissions for
2030 is \hl{13.51}} are interpreted in terms of sum and mean respectively over the relevant data values. Note that it
matters which textual elements are considered to be fixed and which computed: for example, in \textfrag{recorded its
highest emissions in \hl{2030}}, the superlative \emph{highest} is considered fixed, relative to which the year 2030 is
computed via a \kw{maximumBy} query. But we could equally consider 2030 to be fixed, and select the appropriate
superlative (\emph{highest}, or potentially \emph{second-highest}) using an explicit computation of rank. Both sorts of
scenario are represented in our testing data.

We also consider \emph{trend} expressions, which comparative natural language phrases describing how a data
attribute evolves over time, such as \textfrag{training time growing from 67 to 106 seconds}. Such idioms are
mapped to higher-order functions like \kw{trendWord} parameterised on additional helper functions such as
\kw{growShrink} and \kw{betterWorse} (shown in \figref{fluid-scigen}) which map comparisons to appropriate
natural language phrases.

\begin{figure}%[h]
    \small
    {\lstinputlisting[language=Fluid]{../fluid-common/scigen.fld}}
    \caption{Sample of helper functions used in SciGen case study}
    \label{fig:fluid-scigen}
\end{figure}


Taken together, these categories cover a representative portion of the numerical reasoning idioms found in the \SciGen
benchmark. However, some linguistic forms that commonly arise in scholarly articles are not covered in our analysis.
\todo{Move following discussion to Future Work?} We have yet to study approximate quantitative terms like
\textfrag{around 50\%} or \textfrag{roughly 100 instances}, nor interval-based descriptions such as \textfrag{between 30
and 40\%} or \textfrag{within 5–10 seconds}. While we have no reason for thinking these will present specific
difficulties, other forms are likely to be more challenging. So-called \emph{graded} modal adverbs~\citep{lassiter2017}
which modify adjectival comparatives like \textfrag{better} -- as in \textfrag{slightly better} and
\textfrag{significantly higher} -- especially when combined with trends over time, as in \textfrag{steadily increasing}
or \textfrag{sharply declining} -- are likely to prove difficult because the interpretation of these qualifiers can be
subjective and context-dependent. Generalised quantifiers like \textfrag{generally} and
\textfrag{usually}~\citep{barwise1981} present similar challenges because colloquial use may differ from more formal
uses (in some situations ``most'' might mean a majority, i.e.~greater than 50\% of cases, but in others may mean only
``greater than any other alternative proportion''). On the other hand these difficulties also present themselves to
human readers, so extending coverage to these idioms would substantially deepen our tool's ability to bridge natural
language reporting with interpretation in terms of the underlying dataset, perhaps revealing inconsistent use of
technical language on the part of the author. We discuss this further in \Secref{conclusion}.
