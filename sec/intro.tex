\section{Problem Specs}
One of the things currently unclear with respect to this piece of research is the
specification of the problems, and how we are sending that information to an LLM.
There are two things we think we'll want to do:
\begin{enumerate}
   \item \textit{Synthesizing numerical/structural expressions related to data:} if we are referring to a simple aggregate summary of data, we want to be able to synthesize an in-place expression that helps out explanation. For example, if we are explaining a bar-chart: ``the total for the USA, in the year 2015 is \kw{\{totalFor "USA" data2015\}}''. In many cases, we can write this ourselves, but would like to be able to synthesize the expression in brackets
   \item \textit{Synthesizing string expressions based on those numerical expressions:} this is similar to the nombre library in R, once we are able to synthesize references to numerical expressions and data, we want to be able to take those, and use them to create small strings to splice into an explanation. For example ``the \{data for the USA\} is more than \{data for Germany\}'' could be generated by an expression like ``\kw{compare (totalFor "USA" data2015) (compare totalFor "Germany" data2015)}'', where the function \kw{compare} is itself synthesize by the LLM.
\end{enumerate}