\section{Introduction}

When reading a scientific article, the reader's comprehension of the text may
be greatly improved by inclusion of charts and figures that help summarise information.
Visualizations abstract data, in order to make it easier for the reader to understand,
but their comprehension of the information can be aided further by building interactive
features into the visualization. Allowing the user to, for example, click on parts
of a chart to see what data it summarizes allows them to get a better sense of
how the data was used, and what the author is trying to say. By making the text
itself an interactive feature, the readers understanding can be improved yet again.

Authoring of such articles is difficult and time-consuming, and taking an article
and building interactive features into its visualisations requires a further investment
of effort, often on an ad-hoc basis. As yet, there has been no tool that extends
interactivity features to treat text as an interactive visual element. It is not
entirely clear what such features look like when extended to text, and requiring
the author to spend even more time effectively programming their article imposes
a significant burden on the author. 

Some of the difficulty can be mitigated by building visualizations in a language
which provides automatic support for interaction, but these do not provide tools
or support for linking data directly to the text of an article. An article will
be full of references to items of discourse, and asking the author to manually
link each reference to its referent for the purposes of interaction is liable to
error, or the user simply giving up. An automated tool is required. 