\section{Introduction}

When reading a scientific article, the reader's comprehension of the text may
be greatly improved by inclusion of charts and figures that help summarise information.
Visualizations abstract data, in order to make it easier for the reader to understand,
but their comprehension of the information can be aided further by building interactive
features into the visualization. Allowing the user to, for example, click on parts
of a chart to see what data it summarizes allows them to get a better sense of
how the data was used, and what the author is trying to say.

Authoring of scientific articles is difficult and time-consuming. Taking such
an article and building interactive features requires a further investment of
effort, often on an ad-hoc basis. Extending this interactivity to work on text
is even more difficult, making interactive article writing impractical, despite
the clear benefits to transparency, and communication of information.

Some of the difficulty can be mitigated by building visualizations in a language
which provides automatic support for interaction, but these do not provide tools
or support for linking data directly to the text of an article. Scientific articles
will make reference to data consistently.