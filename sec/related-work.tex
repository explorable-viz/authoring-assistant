\section{Related Work}
\label{sec:related}

Our work bears a strong similarity to the notion of explorable explanations \cite{victor11b}. An explorable
explanation as laid out by Victor is an interactive document that allows the reader to explore a quantitative
topic, by manipulating parameters of the underlying model, and observing how these changes impact the results
of the analysis in real time. The crucial difference with our work is that whilst an explorable exploration
allows for the reader to explore a model and its implications, it may not be transparent, in the sense of
exposing the underlying model and its assumptions.

Further work on this line of thought includes \cite{steegen16} who introduced the idea of a multiverse
analysis. Their work is aimed at answering the question of how robust a given data analysis to changes in the
underlying methodology. A multiverse analysis involves taking a given dataset, and applying multiple different
analysis techniques to it, reporting on the difference in results across the various choices. This was then
taken further by \cite{dragicevic19}, wherein they authors build \textit{explorable} multiverse analyses,
allowing the user to toggle between the various methodological choices, and seeing their document update
itself in real time. Our work builds on this, with the constructed multiverse analysis documents also having
their own interactive features within a given set of choices.

Ours is the first work that we are aware of that includes interactive data exploration within a multiverse
analysis, allowing the user to explore intra-universe data depdendency, as opposed to simply surfacing the
changes in results across different choices of method, thus supporting our goal of data transparency .